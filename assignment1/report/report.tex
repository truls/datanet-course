% This is "sig-alternate.tex" V1.9 April 2009
% This file should be compiled with V2.4 of "sig-alternate.cls" April 2009
%
% This has been modified for use in the Principles of Computer System Design @
% DIKU, 2010/2011
\documentclass{sig-alternate}
%
% This gets rid of the copyright box which is not needed for PCSD assignments
%
\makeatletter
\def\@copyrightspace{}
\makeatother
%
% For our purposes page numbers are not so bad
%
\pagenumbering{arabic}

%
% Useful packages
%
\usepackage{url}
\usepackage[english]{babel}
%\usepackage[british]{babel}
\usepackage{hyperref}
%\usepackage{graphicx} % uncomment if you are using graphics

%
% You can delete this pacakge once you start using the template
%
\usepackage{lipsum}

\begin{document}

\title{Datanet, Spring 2011, DIKU\\1st Assignment - HTTP server}

\numberofauthors{1} % This is individual work right???

\author{
% The command \alignauthor (no curly braces needed) should
% precede each author name, affiliation/snail-mail address and
% e-mail address. Additionally, tag each line of
% affiliation/address with \affaddr, and tag the
% e-mail address with \email.
\alignauthor
Truls Asheim\\
       %\affaddr{You don't have to put an address, though you could...}\\
       \email{truls@diku.dk}
}

\maketitle

\begin{abstract}

%
% Put your own text here and delete this comment and the \lipsum command
%
%\lipsum[1]
  This report describes the implementation of a simple concurrent HTTP
  server in erlang. All of the basic problems faced when working with
  HTTP will be exposed because of a restrictive assignment. The
  server as implemented is able to perform basic file-serving functions
  using a limited subset the HTTP protocol.

\end{abstract}

%
% Delete all the text between this comment and the next 
% one, and add your own  text, sections, figures, tables, 
% etc., here...
%
\section{Introduction}

In this assignment we are to implement a simple HTTP server in our
language of choice. I have implemented my server in erlang. Erlang is
a concurrent, functional language designed specifically for
implementing concurrent server applications with an emphasis on
availability and stability.

As HTTP is a huge and in many ways complex protocol only a subset of
it have been implemented in this assignment. It does, however 
implement enough to be able, with certain limitations, to serve files
to HTTP compatible clients.  

Finally, a number of tests has been performed on the server to
ensure that it works as intended when being queried with both correct
and erroneous requests.


\section{Design}
The design of this server is highly influenced by the chosen
implementation language. As it is common in functional languages it is
built around many small well defined functions. This creates readable
programs which are easy to debug and understand. Furthermore it makes
reusing code easier as dependencies within the program are kept to a
minimum. 

Lets look at the lifecycle of the server. This will serve both as an
insight into the ideas behind the design and as a ``What does this
do when and why?'' guide for the code:
\begin{enumerate}
\item Upon invocation of \verb!server:start()! a socket is opened and
  two processes are spawned. \verb!lstn_loop! puts the socket in 
  accepting mode and \verb!mime_lookup! provides a mime lookup service
  used for providing correct MIME-type for the files served.
\item When \verb!lstn_loop! receives a connection it spawns another
  instance of itself and calls \verb!rcv_loop!. So basically the
  process which receives the connection will die when the request has
  been processed.
\item Once the HTTP request has been received it is processed by the
  body of \verb!rcv_loop!. Here, the headers is parsed and request
  type, uri and other information is extracted. If the request type is
  found to be supported the rest of the request will be processed.
\item \verb!handle_path! its then called and receives the URI, request
  type and headers as parameters. Based on this it can take one of
  three paths.
\begin{enumerate}
\item If the URI refers to a file, the file is opened and
  returned. The HTTP response is assembled by the function
  \verb!http_respones!. The process \verb!mime_lookup! is asked to map
  the extension of the file to a MIME-type.
\item If the URI refers to a directory, a directory listing is
  generated and returned.
\item If the URI does not exists or is inaccessible an error is
  returned. All errors are generated by the \verb!http_error! which generates
  a complete response based on a HTTP status code.
\end{enumerate}

All responses which consist of HTML output is passed through the
function \verb!html_function! which acts as a template.

\item Finally, the response is sent back to the client and the socket
  is closed.

\end{enumerate}

The server's ability to handle multiple requests simultaneously is
introduced by the actions in step two. Once the listening process
receives a request it spawns another instancs into a process which
processes the request received. The newly spawned process replaces the
listening role the old process once had.

\section{HTTP implemented}

The purpose of this section is to account for the subset of HTTP
implemented. HTTP is a huge protocol so supporting everything is not
feasible This section is dedicated to the parts of the HTTP protocol
that is implemented.


\subsection{Headers}
\subsubsection{Request}
Currently, none of the request headers are taken into account when
processing a request. This is in part due to the fact that most of
them are irrelevant to the servers prime purpose which is to serve raw
files. This section will therefore be dedicated to describing the most
important headers which are present in a request and what their
purpose is. The decision to avoid taking headers into account was also
made because some headers, Accept* especially, is rather complicated
to parse.

The infrastructure to take headers into account when generating a
response is present in the server so doing this will be easy whenever
needed by future expansions.

\begin{description}
\item[Accept] This header contains the client's preferred
  MIME-types. Abiding this header is not mandatory according to
  section 14.1 of the RFC.
\item[Accept-Encoding] This is perhaps one of the most important
  headers in the request. It can be used by the server to ensure that
  responses are sent in a character set which is understood by the
  client.
\item[Connection] indicates the clients preferred connection type. Its
  values are keep-alive and close. This server will always send back
  ``Connection: close'' to instruct the client to close the connection
  right after the response have been received.
\item[Host] This header is fundamental to most of the internet as it
  allows servers to use virtual hosting. The Host header contains the
  hostname used in the request URL and is used by servers to determine
  which site should be sent.

\end{description}

\subsubsection{Response}

This section contains a description of the headers included in a
response from the server and their relation to the RFC.

\begin{description}
\item[Connection] Connection: closed is sent with every response. This
  is required by RFC2616 section x.x. The value closed requires the
  client to close the connection immediately after the request have
  been received.
\item[Date] The date field is populated with a timestamp in RFC1123
  format during header compilation in the server. The presence of this
  header and usage of RFC1123 date format is required by RFC2616
  section 14.18
\item[Last-Modified] is sent only if the request is for a file. Its
  value is taken from the file system. Its presence is required when
  we get to implementing cache functionality.
\item[Content-Length] Holds the number of octets in the response
  body. This is taken from the filesize when sending a file or by
  counting the number of bytes in the response when returning a
  filelisting.
\item[Content-Type] is sent with every response. If the response is a
  directory index or a error message this is set to
  \verb!text/html!. If the request if for a file its extension is used
  to look up a MIME-type in \verb!/etc/mime.types!

\end{description}

\subsection{Limitations}
The server only accepts requests from HTTP/1.1 clients. Requests made
in other versions of the HTTP protocols will be rejected with a 505
error code. Despite of this being in violation with RFC2616 this
decision was made to avoid dealing with the complexities introduced by
backwards compatibility. For instance, HTTP/1.1 compatible servers and
clients are required by RFC2616 section 3.3 to handle three date
formats in order to maintain backwards compatibility with HTTP/1.0
clients.

It doesn't implement chunked coding. In order to conform to RFC2616
section 3.6.1 a server must be able receive requests transmitted in
chunked encoding even if it is unable to send them. Implementing this
was deemed to be beyond the scope of this assignment.

The server is also somewhat lacking in the area of error handling. If
the server crashes while handling a request the TCP socket is closed
immediately without notice. If this was handled correctly, the client
would receive a 500 status code.

\section{Extensibility}

One of the requirements of this assignment was that the design of the
implemented server should support future expansion without requiring
major changes. This server does indeed fulfill this requirement given
its modular design. All of the basic functions for handling the HTTP
protocol are present and can be reused immediately. Furthermore, file
handling is limited to a single function which means that the server
can be made to serve any content simply by changing this one
function. The functions for parsing HTTP request are also flexible
enough to support currently unsupported HTTP methods and non-standard
headers.

In the next assignment we are to implement a caching HTTP proxy
server. One obvious way to transform this server into a proxy would be
to simply replace the file handling code with a datastore and a HTTP
client. Since we are working in erlang an obvious choice for at
datastore would be ETS. ETS is erlang's native key/value store capable
of handling large amounts and can which can exist both in memory or on
disk.

Of course not all functionality required by a proxy is
present. For instance, cache-related headers are currently not taken into
consideration when processing a request. This will need to be added
along with code for doing date processing and conversion. The HTTP
request header parser also makes no distinctions based on the semantics
of the request URL. This is also required by a working proxy
server. All of this can however, be implemented within the limits of
the design.

\section{Testing}
Below is a list of all the tests performed together with an indication
of whether or not the operation succeeded with expected results. The
tests have been executed manually using a combination of
\verb!curl(1)! and Firefox 4 with Firebug. ``White Box'' testing has
been used, meaning that only results and error conditions which are
either known to the author or deliberately emitted by the code are
exposed. As noted earlier, should an unexpected critical condition
arise the socket will be closed immediately without any response being
sent.

\begin{itemize}
\item GET requests for existing files: As expected, The file is
  returned
\item GET requests for existing folders: As expected: A list of files
  in the folder is returned.
\item GET requests for non-existing resource: As expected, A 404 code is
returned together with a HTML error page
\item GET requests for files/folders without read access: As expected, 403
is returned

\item HEAD successful requests: As expected. The headers returned by a HEAD
request are identical to those returned by a GET request.
\item HEAD failed request: \textbf{FAIL} (In violation), A full reply with a body
  containing an error message. According to RFC2616 section 9.4 a HEAD
  request must \textbf{never} include a body.

\item Request with invalid request line (for instance, ``GET/ FOOTTP/a.b''): As expected, 400 is returned
\item Request method other than GET or HEAD: As expected: 501 is returned
\item Request contains invalid header (for instance, header contains a
  line which does not match the format ``Header-Field: header value'': As expected, 400 is returned
\end{itemize}

%
% Delete all the text between this comment and the previous
% one, and add your own text, sections, figures, tables, 
% etc., here...
%


%\bibliographystyle{abbrv}
%\bibliography{pcsd}  % pcsd.bib is the name of the Bibliography in this case
% You must have a proper ".bib" file
%  and remember to run:
% latex bibtex latex latex
%  or
% pdflatex bibtexx pdflatex pdflatex
% to resolve all references

%APPENDICES are optional
%\balancecolumns
%\appendix
%Appendix A
%\section{Headings in Appendices}

% That's all folks!
\end{document}
