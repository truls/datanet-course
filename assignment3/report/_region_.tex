\message{ !name(report.tex)}% This is "sig-alternate.tex" V1.9 April 2009
% This file should be compiled with V2.4 of "sig-alternate.cls" April 2009
%
% This has been modified for use in the Principles of Computer System Design @
% DIKU, 2010/2011
\documentclass{sig-alternate}
%\documentclass{article}
%
% This gets rid of the copyright box which is not needed for PCSD assignments
%
\makeatletter
\def\@copyrightspace{}
\makeatother
%
% For our purposes page numbers are not so bad
%
\pagenumbering{arabic}

%
% Useful packages
%
\usepackage{url}
\usepackage[english]{babel}
%\usepackage[british]{babel}
\usepackage{hyperref}
%\usepackage{graphicx} % uncomment if you are using graphics

%
% You can delete this pacakge once you start using the template
%
\usepackage{lipsum}

\begin{document}

\message{ !name(report.tex) !offset(-3) }


\title{Datanet, Spring 2011, DIKU\\3rd Assignment - Anonymizing
  Distributed Proxy}

\numberofauthors{1} % This is individual work right???

\author{
% The command \alignauthor (no curly braces needed) should
% precede each author name, affiliation/snail-mail address and
% e-mail address. Additionally, tag each line of
% affiliation/address with \affaddr, and tag the
% e-mail address with \email.
\alignauthor
Truls Asheim\\
       %\affaddr{You don't have to put an address, though you could...}\\
       \email{truls@diku.dk}
}

\maketitle

\begin{abstract}

%
% Put your own text here and delete this comment and the \lipsum command
%
%\lipsum[1]
  This report describes the implementation of and various
  considerations regarding a HTTP proxy server which has the ability
  to operate as a part of an anonymizing distributed network which
  routes each request through a number of nodes masking its
  origin. Both the server itself and various problems related to the
  protocol used by the network will be covered.

\end{abstract}

%
% Delete all the text between this comment and the next 
% one, and add your own  text, sections, figures, tables, 
% etc., here...
%

\section{Design}
The design of this server is organized as an Erlang OTP supervision
tree where each major component of the server is monitored by
supervisors which takes care of restarting the servers if anything
goes wrong. This is consistent with the Erlang philosophy of failing
early and starting over as opposed to attempting to do possibly
incomplete error handling.

The major components of the server are:
\begin{description}
\item[prtracker] This server is responsible maintaining the domain
    whitelist, the nodelist, and the node blacklist. It also
    registers the proxy against the tracker periodically.
\item[prnodeparser] contains the code for parsing the pure-text
  version of the peerlist. It receives request from \verb!pr_tracker!
  whenver a new version of the trackerlist needs parsing.
\item[prsup] is a supervisor which is responsible for a number of
  \verb!pr_servers! (see below). It spawns a new \verb!pr_server! for
  every request received.
\item[prrootsup] This is the supervisor of all the previously
  mentioned components. It makes sure that they are running at all
  times and is responsible for restarting them if they crash.
\item[prserver] This is the main component of the server which is
  responsible for processing
\end{description}

The packaging of the server is quit messy. This will hopefully be
fixed in the next assignment. See the file \verb!README! in the folder
\verb!datanetproxy-0.1! for information on how to compile and run the
code.

%For assignment 4: This is once again a design description even though
%the design hasn't changed significantly since assingment 3 the design
%description in assignment 3 was lackluster to say the least and I
%want to make up for it by providing a proper description of it in
%this final assignment.

\section{Setup}
I am running my proxy on my own server hosted by Hetzner online in
Germany. It is connected to a to the outside world through a 100mbit
uplink. It has ports open to the outside world.

\section{Protocol Limitations}
The current tracker protocol has two (in my opinion) major flaws:
\begin{enumerate}
\item The tracker keeps peers listed for way to long. This causes
  peers which are no longer active to remain in the list for a long
  time requiring each peer to keep extensive blacklists containing
  peers which for various reasons aren't responding correctly. The
  tracker should expire peers shortly after the minimum registration
  interval.
\item The fact that the tracker locks you out if you register more
  frequently than the min-wait interval can be problematic in many
  circumstances.
\end{enumerate}

Furthermore, the general implementation quality of the peers in the
network seems to be quite low causing a high fail-rate for requests
and slow transfers. One way of combating this could be to implement a
distributed blacklist where the peers could use the registration process
to provide the tracker with a list of newly discovered bad peers.

I have also observed a few peers which persistently produces erroneous
responses which on the surface seems correct. For instance, I observed
one peer which responded to every request with a HTTP 500 (Internal
Server Error) status code. The problem with these kinds of responses
is that they, on the surface, looks correct. Continuing our previous
example it is hard to differentiate a HTTP 500 response produced by a
rouge proxy and a ``legitimate'' HTTP 500 error produced by the
destination host. As previously suggested one way of handling this
problem could be to implement a distributed 

One major annoyance when working with the protocol was that the
tracker holds on to nodes for way to long

\section{Misbehaving peers}
As noted in the previous section, misbehaving peers is a major problem
in the network. In order to have any hope of providing a reasonable
performance each peer must maintain it's own blacklist. I use the
following method in order to maintain a blacklist of peers which
should be avoided:

\begin{itemize}
\item Whenever a peer causes a connection timeout (see below) it is
  added to the blacklist and removed from the list of working peers
  and added to the blacklist
\item Whenever the \verb!pr_tracker! components updates the list of
  trackers it makes sure that no blacklisted peer reenters the list of
  peers \textit{unless} the peer has registered with the tracker since
  it was blacklisted.
\item After a new peerlist has been downloaded from the tracker it is
  checked against the blacklist to ensure that peers which have been
  removed by the tracker also gets removed from the blacklist.
\end{itemize}

There are two kinds of timeouts; connection timeouts and response
timeouts. Peers which causes timeouts by refusing connections should
obviously be blacklisted. The peers which produces response timeouts
are more of a gray area which is primarily divided into two groups:

\begin{enumerate}
\item Peers which accepts a connection and then never responds due to
  an error in peer and
\item Peers which are in fact functioning but gets unlucky and chooses
  a non-reponding peer as it's next hop.
\end{enumerate}

The first group should be blacklisted, but doing so with the second
group would be unfair. In my implementation I have therefore decided
to not blacklist any of these peers and instead send back a 504
``Gateway timeout'' error back to the client. One way of easing differentiation
between these two groups would be to implement mandatory timeout
values across the network which shouldn't be exceeded by any request.

\section{Security Measurements}
The proxy server is only willing to process requests for domains
listed in the whitelist as provided by the tracker. This is ensured by
performing a check on every request domain making sure that it is
listed in the whitelist. If the proxy receives a request for a domain
which is not in the whitelist it will respond with a 304 ``Forbidden''
error and a custom error page.

\section{Testing}

All tests has been performed with the \verb!Max-Forward! header set to
3. The proxy server automatically enforces this for all requests
lacking the header.

\subsection{General browsing}
These times have been found using Firebug with the extension YSlow
which shows the total load-time of a webpage. My proxy is currently
unable to handle pages with gzip encoding and therefore not all pages
in the whitelist are browsable. All the tests below has been performed
on the main pages of the respective domains.
\\
\\

\begin{tabular}{|l|c|c|}
\hline
\textbf{Domain} & \textbf{Times tested} & \textbf{Average} \\
\hline
www.slashdot.org & 3 & 100.318s\\
\hline
www.reddit.com & 3 & 114s\\
\hline
www.wikipedia.org & 3 & 59.717s \\
\hline
www.vimeo.com & 3 & 77.07s\\

\hline
www.xkcd.com & 3 & 66.155s\\
\hline
\end{tabular}

\subsection{Bandwidth}
The bandwidth tests has been performed using \verb!curl(1)!. The
reported average is the average of the average transfer speeds
reported by \verb!curl(1)!. The following command was used for the
benchmark: 

\begin{verbatim}
http_proxy=213.239.205.47:8000 curl \
http://upload.wikimedia.org/wikipedia/\
commons/5/52/Edit_01-12-09_small.ogg > /tmp/foo
\end{verbatim}

The test results for the bandwidth test are included below
\\
\\
\begin{tabular}{|c|c|c|}
\hline
\textbf{Run no.} & \textbf{Time} & \textbf{Avg  speed} \\
\hline
1 & 226s & 137k  \\
\hline
2 & 499s & 63.8k \\
\hline
3 & 106s & 291k \\
\hline
4 & 62s  & 495k \\
\hline
\textbf{Averages} & 225.25s & 246.7 \\
\hline
\end{tabular}



%
% Delete all the text between this comment and the previous
% one, and add your own text, sections, figures, tables, 
% etc., here...
%


%\bibliographystyle{abbrv}
%\bibliography{pcsd}  % pcsd.bib is the name of the Bibliography in this cas63815e
% You must have a proper ".bib" file
%  and remember to run:
% latex bibtex latex latex
%  or
% pdflatex bibtexx pdflatex pdflatex
% to resolve all references

%APPENDICES are optional
%\balancecolumns
%\appendix
%Appendix A
%\section{Headings in Appendices}

% That's all folks!
\end{document}

\message{ !name(report.tex) !offset(-290) }
